\documentclass[12pt]{article}

\usepackage[english]{babel}
\usepackage[letterpaper,top=3.5cm,bottom=3.5cm,left=3cm,right=3cm]{geometry}
\usepackage{import}
\usepackage{amsmath}
\usepackage{graphicx}
\usepackage[svgnames]{xcolor}
\usepackage[colorlinks=true, allcolors=blue]{hyperref}
\usepackage[backend=biber, style=chem-angew]{biblatex}
\usepackage{listings}

\graphicspath{{./immagini/}}

\lstset{
    language=R,
    showstringspaces=false,
    columns=flexible,
    basicstyle={\scriptsize\ttfamily},
    numbers=left,
    numberstyle=\tiny\color{gray},
    stepnumber=1,
    stringstyle=\color{Red},
    commentstyle=\color{DarkGreen},
    breaklines=true,
    breakatwhitespace=true
  }


\addbibresource{bibliography.bib}

\title{
    Tesina sul pacchetto StatsBombR \\ 
    \vspace{5pt} \large{football analytics in R} \\ 
    \vspace{20pt} \normalsize{Corso di Statistica Applicata e Analisi dei Dati a.a. 2022/23} \\ 
    \vspace{10pt} \textit{Zamolo Riccardo, Mat. 135047}
}

\begin{document}
    \date{}
    \maketitle
    
    \newpage
    
    \listoffigures
    \tableofcontents

    \newpage

    \import{./}{parte1.tex}

    \newpage

    \import{./}{parte2.tex}

    \newpage
    
    \appendix
    \section{Codice esempi}
    \lstinputlisting{./codice/esempio1.r}
    \vspace{30pt}
    \lstinputlisting{./codice/esempio2.r}
    \vspace{30pt}
    \lstinputlisting{./codice/esempio3.r}
    \newpage
    
    \printbibliography
\end{document}
