\section{Introduzione}
            La seguente tesina ha lo scopo di presentare una descrizione dell'argomento scelto, ovvero un approfondimento sull'utilizzo del pacchetto R "StatsBombR". Il contenuto della tesina, si può pensare, a livello logico, essere suddiviso in due parti: una prima parte di descrizione teorica del pacchetto e una seconda parte in cui si mostrano alcuni esempi pratici, svolti con gli strumenti offerti. 

            Per la presentazione della seconda parte, si assume che il lettore abbia familiarità con le regole basilari del calcio. Una breve descrizione verrà comunque fornita quando vi sarà il bisogno di introdurre un concetto legato a tale sport. La scelta del tema per la tesina è legata a un interesse personale verso lo sport, in particolare al calcio, avendolo personalmente praticato per anni. Inoltre, fin da subito, mi hanno colpito le applicazioni possibili che si possono fare combinando le funzioni del pacchetto.

            Le principali fonti utilizzate nel presente documento, sono le stesse allegate nella traccia, ovvero le slide di presentazione del pacchetto \cite{slideSBR}, la pagina web di StatsBomb \cite{StatsBomb} e la repository Github \cite{FCrSTATSGithub}, relativa alla football analytics. Altre fonti aggiuntive utilizzate, si possono trovare sotto forma di lista alla fine del documento.


\section{StatsBombR - Descrizione teorica}
    Come detto precedentemente, nella presente sezione, si vuole fornire una descrizione teorica del pacchetto R "StatsBombR", contestualizzando inizialmente le football analytics. Inoltre si elencheranno le dipendenze necessarie e i pacchetti opzionali utilizzati nella seconda parte che possono essere combinati con StatsBombR.

    \subsection{Football analytics}
        Prima di illustrare le caratteristiche del pacchetto, si vuole dare un'idea di che cosa siano le football analytics. Nel definire brevemente le football analytics, si farà riferimento a un concetto più generale, ovvero le sport analytics. 
        
        Per analisi sportive, come riportato nella pagina Wikipedia \cite{SportAnalytics}, si intende una raccolta di statistiche storiche e/o attuali, capaci di fornire un vantaggio competitivo a una squadra o a un singolo individuo. L'idea quindi è di apprendere quante più informazioni possibili dagli avversari, al fine di rendere il processo decisionale (movimenti, giocate e scelte) più facile, sia prima che durante la gara. Non solo, il fine è anche quello di raccogliere e utilizzare i dati per migliorare la qualità degli allenamenti e, di conseguenza, le performance. 
        
        La pagina Wikipedia riporta il famoso film "Moneyball", grazie al quale le analisi sportive hanno cominciato a prendere sempre più piede, in quanto il General Manager della squadra di baseball "Oakland Athletics", sfrutta le analisi per costruire una squadra competitiva con un budget minimo. Tornando alle analisi calcistiche, si può certamente affermare che tale sport, al giorno d'oggi è in continua evoluzione e ad esempio, agli ultimi mondiali in Qatar (2022) ha debuttato il "semi-automated offside", una nuova tecnologia in grado di supportare l'arbitro nello stabilire la giusta decisione riguardo a situazioni di difficile interpretazione, durante una partita. La tecnologia consiste in dodici telecamere installate nello stadio che tracciano 29 punti tridimensionali di ogni singolo giocatore, indicando con ancora maggior accuratezza la sua posizione in campo. Maggiori informazioni si possono trovare sul sito della FIFA \cite{Semi-auto-offside}. Non solo, un'azienda della zona mette a disposizione dei dispositivi indossabili in grado di produrre dati relativi alle performance degli atleti durante gli allenamenti, in maniera che le squadre, ovvero i clienti, possano trarre benefici multipli, come ad esempio la calibrazione del carico di lavoro di un atleta durante un periodo di recupero da un infortunio, migliorare le perfomance degli allenamenti oppure ridurre il rischio di eventuali infortuni. Ulteriori informazioni si possono trovare direttamente qui \cite{gpexe}.

        Nella digressione precedente, si è cercato, in poche righe, di esprimere l'idea che nel calcio e in generale negli sport, al giorno d'oggi, la tecnologia ma soprattutto i dati giocano un ruolo molto importante ed è quindi molto vantaggioso disporre di molti dati per poter effettuare delle analisi che sicuramente porterebbero numerosi vantaggi. Nella prossima sotto-sezione, si vuole introdurre il pacchetto "StatsBombR" e, in generale, che tipo di analisi si possono fare con gli strumenti che offre.
    
    \subsection{StatsBombR}
        StatsBombR è un pacchetto R, sviluppato dal data scientist Derrick Yam, il quale lavora per la società StatsBomb, specializzata nella diffusione di dati legati al calcio. I dati che la società offre, sono principalmente rivolti ad analisti, persone che lavorano in ambito legato alla statistica e a creatori di contenuti multimediali (media creators). Il pacchetto R StatsBombR è scaricabile dal repository GitHub \cite{StatsBombRGithub}.
        
        A questo punto, si vuole evitare di proporre un listato di tutte le funzioni R del pacchetto, disponibili qui \cite{funzioniR}, in quanto risulterebbe noioso e fine a se stesso. Si vuole procedere illustrando brevemente le dipendenze di StatsBombR e i pacchetti opzionali che sono stati utilizzati nella seconda parte, dove al bisogno, verrà fornita una spiegazione delle funzioni man mano che verranno utilizzate.
        
        In ogni caso, va fatta un'importante distinzione tra dati ad accesso libero e dati a pagamento, resi disponibili dalla società StatsBomb tramite sottoscrizione. Il seguente testo si focalizza solamente sui dati ad accesso libero, quindi i dati disponibili sono in numero limitato. Lo stesso discorso è applicabile alle funzioni richiamabili del pacchetto StatsBombR, dal momento che molte necessitano di un nome utente e una password come parametri d'ingresso. Come detto poco sopra, nella prossima sotto-sezione, verranno presentate le dipendenze di StatsBombR, con l'idea di illustrare brevemente le componenti necessarie per utilizzare al meglio il pacchetto e, contemporaneamente semplificare, nella seconda parte del testo, la spiegazione delle funzioni utilizzate.
        
    \subsection{Dipendenze}
        Nella presente sotto-sezione, l'obiettivo è quello di presentare una breve carrellata delle dipendenze di StatsBombR, ovvero siano, gli archivi esterni da cui StatsBombR dipende. Lo studio delle dipendenze è quindi sorta di base di partenza per comprendere al meglio, nella seconda parte, le features di StatsBombR. Infine, vengono presentati anche altri pacchetti opzionali utili utilizzati e che si possono utilizzare.

        \subsubsection{Pacchetti necessari}
        \begin{itemize}
            \item \textbf{tidyverse}: è un insieme di pacchetti R che mette a disposizione funzioni per gestire i dati, quali la manipolazione, la visualizzazione, l'importazione e il riordino. Prevede anche molte altre funzioni, per esempio relative alla programmazione funzionale. Per l'installazione digitare il comando sulla console \\ \texttt{install.packages("tidyverse")}
            
            \item \textbf{ggplot2}: è uno dei pacchetti più famosi e importanti per la visualizzazione dei dati in R. Il suo funzionamento è molto semplice: Si forniscono i dati in ingresso e si specifica alla funzione di riferimento come i dati devono essere rappresentati graficamente (e quindi quali primitive usare). ggplot2 è contenuto in tidyverse e quindi non richiede alcuna installazione particolare.
            
            \item \textbf{devtools}: Molti pacchetti sono scaricabili direttamente dall'archivio ufficiale di R \cite{CRAN}, altri sono scaricabili sia da GitHub che dall'archivio ufficiale di R. Altri ancora, tuttavia, sono scaricabili solamente da GitHub. StatsBombR rientra in quest'ultima opzione e quindi è necessario installare il pacchetto "devtools" che consente di scaricare pacchetti da GitHub. Per l'installazione digitare il comando sulla console \texttt{ install.packages("devtools")}

            \item \textbf{StatsBombR}: Il pacchetto con cui si reperiscono i dati e si effettueranno, nella seconda parte della presente relazione, gli esperimenti sui dati. Per l'installazione digitare il comando sulla console \\ \texttt{devtools::install\_github("statsbomb/StatsBombR")}
        \end{itemize}
        
        \subsubsection{Ulteriori pacchetti utili}
            \begin{itemize}
                \item \textbf{ggsoccer}: il pacchetto fornisce funzioni utili per plottare in maniera alternativa i dati legati agli eventi che succedono in una gara di calcio. Per l'installazione digitare il comando sulla console \texttt{install.packages("ggsoccer")}
                \item \textbf{soccermatics}: soccermatics offre strumenti per visualizzare graficamente dati spaziali calcistici ed dati legati ad eventi, ad esempio heatmaps, traiettorie individuali di giocatori e coordinate x,y per plotting/metriche. Per l'installazione digitare il comando sulla console \texttt{devtools::install\_github("jogall/soccermatics")}
                \item \textbf{ggrepel}: fornisce informazioni geometriche per il pacchetto ggplot2. Utile quando si hanno problemi di sovrapposizione tra etichette in un grafico. Per l'installazione digitare il comando sulla console \texttt{devtools::install\_github("slowkow/ggrepel")}
                \item \textbf{gganimate}: estende ggplot2, in quanto permette di creare grafici più complessi e animati. Per l'installazione digitare il comando sulla console \\ \texttt{devtools::install\_github("thomasp85/gganimate"}
                \item \textbf{FCrSTATS}: un pacchetto che contiene a sua volta pacchetti e funzioni utili per le football analysis e la visualizzazione dei dati. Per l'installazione digitare il comando sulla console \texttt{devtools::install\_github("FCrSTATS")}
                \item \textbf{ggtern}: il pacchetto può essere considerato un'estensione di ggplot2, in quanto consente di creare i cosidetti "ternary diagrams". Per l'installazione digitare il comando sulla console \texttt{install.packages("ggsoccer")}
            \end{itemize}